\documentclass[journal,12pt,twocolumn]{IEEEtran}
\usepackage[utf8]{inputenc}
\usepackage[margin=1.25in]{geometry}
\usepackage{amsmath,amssymb}
\usepackage{tfrupee}
\usepackage{graphicx}
\usepackage{tabularx}
\title{Assignment1}
\author{G Harsha Vardhan Reddy CS21BTECH11017}
\date{March 2022}
\begin{document}
%\begin{math}
\maketitle
\section*{\textbf{ICSE 2018 5(b)}}
\begin{table}[h]
\scalebox{1.18}{
\begin{tabular}{|c|c|c|}
\hline
    \textbf{variable} &\textbf{ symbol }&\textbf{ formula}  \\
    \hline\hline
    Total investment & T & -\\
    \hline
    Value of each share & CP & -\\
    \hline
    Discount & d & -\\
    \hline
    Market Price & MP & CP - d\\
    \hline 
    No.of shares & n & $\frac{\text{T}}{\text{MP}}$ \\[0.5ex]
    \hline
    dividend per share & $d_1$ & - \\
    \hline
    Total dividend & D & $d_1\times n$\\
    \hline
    Rate of return & r & $\frac{D}{T}$\\ [0.5ex]
    \hline
\end{tabular}}
\end{table}
Given\\
Total investment\((T)\) = \rupee 22,500.\\
The value of each share \( (CP) \) = \rupee 50\\
Discount on each share \( (d)\) = 10\%  \[\implies d =\frac{10}{100}\times50\]
 \[\implies d = 5\]
$ \therefore$ discount = \rupee 5 on each share\\
$\therefore$The Market price of each share\((MP)\) = \( CP - d \)\[\implies MP = 50 - 5 \]
$\therefore$ Market Price \((MP)\) = \rupee 45 . 
\begin{enumerate}
    \item 
    Total number of shares purchased \((n)\) \[ n =\frac{CP}{MP}\]
    \[\implies n = \frac{22500}{45}\]
         \[\implies n = 500\]
    $\therefore$On total 500 shares were purchased.
    \item
    Given\\
    Dividend paid by the company \((D)\) = 12\%\\
    Dividend on each share \((d_1)\) = 12\% of \((CP)\)  
    \[\implies d_1 = \frac{12}{100}\times 50 \]
    \[\implies d_1 = 6\]
    $\therefore$ Dividend on each share \((d_1)\) = \rupee 6\\
   Total dividend\((D)\) = $ d_1 \times n $
  \[\implies D = 6 \times 500\]
  \[\implies D = 3000.\]
    $\therefore$Total dividend paid by the company \((D)\) = \rupee3000.
    \item
    Rate of return he gets on investment
    \[ (r) = \frac{D}{T} \times 100\]
    \[\implies r = \frac{3000}{22500}\times 100\]
    \[\implies r = 13.33\% \]
    \[\implies r \approx 13\%\]
    
    $\therefore$ He gets 13\% return on his investment.
\end{enumerate}

%\end{math}
\end{document}
