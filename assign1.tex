\documentclass{article}
\usepackage[utf8]{inputenc}
\usepackage[margin=1.25in]{geometry}
\usepackage{amsmath,amssymb}
\usepackage{tfrupee}
\title{Assignment1}
\author{G Harsha Vardhan Reddy CS21BTECH11017}
\date{March 2022}

\begin{document}
%\begin{math}
\maketitle
\section*{ICSE 2018 Problem 5(b)}

Given\\
Total investment = \rupee 22,500.\\
The value of each share = \rupee 50\\
Discount on each share = 10\%  \[=\frac{10}{100}\times50\]
\[ = 5.\]
$ \therefore$ discount = \rupee 5 on each share\\
$\therefore$The Market price of each share = value of each share - discount =  \rupee 50 - \rupee 5 = \rupee 45
\begin{enumerate}
    \item 
    Total number of shares purchased =\[\frac{\text{Total Investment}}{\text{M.P of each share}}\]
    \[ = \frac{22500}{45}\]
         \[ = 500\]
    $\therefore$On total 500 shares were purchased.
    \item
    Given\\
    Dividend paid by the company = 12\%\\
    Dividend on each share = 12\% of share value  \[ = \frac{12}{100}\times 50 \]
    \[= 6\]
    $\therefore$ Dividend on each share = \rupee 6\\
   Total dividend = Dividend on each share $\times$ Total no.of shares
  \[ = 6 \times 500 = 3000.\]
    $\therefore$Total dividend paid by the company = \rupee3000.
    \item
    Rate of return he gets on investment = \[\frac{\text{Total dividend}}{\text{Total investment}} \times 100\]
    \[ = \frac{3000}{22500}\times 100\]
    \[ = 13.33\% \]
    \[\approx 13\%\]
    
    $\therefore$ He gets 13\% return on his investment.
\end{enumerate}

%\end{math}
\end{document}
