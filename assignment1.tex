\documentclass{article}
\usepackage[utf8]{inputenc}
\usepackage[margin=1.25in]{geometry}
\usepackage{amsmath,amssymb}

\title{assignment1}
\author{G Harsha Vardhan Reddy}
\date{March 2022}

\begin{document}
%\begin{math}
\maketitle
\section*{ICSE 2018 Problem 5(b)}

Given\\
Total investment = Rs.22,500.\\
The value of each share = Rs.50\\
Discount on each share = 10\%  \[=\frac{10}{100}\times50 = 5.\]\\
$ \therefore$ discount = Rs.5 on each share\\
$\therefore$The selling price of each share = Rs.50 - Rs.5 = Rs.45
\begin{enumerate}
    \item 
    Total number of shares purchased =\[\frac{\text{Total Investment}}{\text{S.P of each share}} = \frac{22500}{45} = 500\]
    $\therefore$On total 500 shares were purchased.
    \item
    Given\\
    Dividend paid by the company = 12\%\\
   Total dividend = \[\frac{12}{100} \times 50 \times 500 = 3000.\]
    $\therefore$Total dividend paid by company = Rs.3000.
    \item
    Rate of return he gets on investment = \[\frac{\text{Total dividend}}{\text{Total investment}} \times 100 = \frac{3000}{22500}\times 100 = 13.33\% \approx 13\%\]
    
    $\therefore$ He gets 13\% return on his investment.
\end{enumerate}

%\end{math}
\end{document}
